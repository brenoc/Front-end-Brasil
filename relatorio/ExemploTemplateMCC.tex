\documentclass[11pt,a4paper]{article}
\usepackage{mccpuc}
\usepackage[dvips]{graphicx}
\usepackage[utf8]{inputenc}
\usepackage[portuges,brazilian]{babel}
\usepackage{cite}
\usepackage[dvipdfm]{hyperref}


\begin{document}

%% Preenchimento dos parâmetros da MCC
\TextoPortugues{true} % 'true' para português, 'false' para inglês 
\ISSNno{0103-9741} % Texto do ISSN
\MCCSeqAno{02/10}  % Número e ano(YY) da MCC
\TituloCapa{Exemplo de utilização de um\\ Pacote de Estilo \LaTeX \ para MCC}
\TituloRosto{Exemplo de utilização de um Pacote de Estilo \LaTeX \ para MCC}
\MesTexto{Mês/Month} % Mês por extenso, em português ou inglês
\AnoYYYY{2010}   % Ano com 4 digitos
\QtdLinhasTituloCapa{2} % Qtd de linhas do titulo na capa, de 1 até 4 linhas

\QtdAutores{3} % Define quantidade de autores: 1, 2, ou 3
%%%% Dados Autor 1 %%%%%%%%%%%%%%%%%%%%%%%%%%%%%%%%%%%%%%%%
\AutorANome{Autor 1 (nome por extenso)}
\AutorAemail{autor1@inf.ufrgs.br}
\AutorADI{false} % 'true' se o autor for do DI
\AutorAinst{ % Instituição do autor, caso não seja do DI
Instituto de Informática – Universidade Federal do Rio Grande do Sul (UFRGS)}

%%%% Dados Autor 2 %%%%%%%%%%%%%%%%%%%%%%%%%%%%%%%%%%%%%%%%
\AutorBNome{Autor 2 (nome por extenso)}
\AutorBemail{autor2@inf.puc-rio.br}
\AutorBDI{true} % 'true' se o autor for do DI
\AutorBinst{ % Instituição do autor, caso não seja do DI
}

%%%% Dados Autor 3 %%%%%%%%%%%%%%%%%%%%%%%%%%%%%%%%%%%%%%%%
\AutorCNome{Autor 3 (nome por extenso)}
\AutorCemail{autor3@cma.stanford.edu}
\AutorCDI{false} % 'true' se o autor for do DI
\AutorCinst{ % Instituição do autor, caso não seja do DI
Center for Computer Research in Music and Acoustic, Stanford University, USA}


\PatrocinioFlag{true} % 'true' para incluir nota de rodapé da folha de rosto
\Patrocinio{ % Texto da nota de rodapé para patrocínio
(este, ou outro, quando for aplicável) Trabalho patrocinado pelo Ministério de
Ciência e Tecnologia da Presidência da República Federativa do Brasil (e agência
de fomento e o número do processo, se aplicável). (Em Inglês: This work has been
sponsored by the Ministério de Ciência e Tecnologia da Presidência da República
Federativa do Brasil
}


\Resumo{  % Resumo em Português. Para Trabalho em Português
Resumo em português, resumo em português, resumo em português, resumo em
português, resumo em português, resumo em português, resumo em português, resumo
em português, resumo em português, resumo em português, resumo em português,
resumo em português, resumo em português, resumo em português, resumo em
português, resumo em português, resumo em português, resumo em português, resumo
em português, resumo em português, resumo em português, resumo em português,
resumo em português, resumo em português, resumo em português, resumo em
português, resumo em português, resumo em português, resumo em português, resumo
em português, resumo em português. 
}

\PalavrasChave{ % Palavras-Chave em português
Selecione até cinco palavras-chave que definam precisamente o con\-te\-ú\-do do
seu trabalho.
}

\Abstract{ % Resumo em Inglês. Para trabalhos em Português ou Inglês
Resumo em inglês, resumo em inglês, resumo em inglês, resumo em inglês, resumo
em inglês, resumo em inglês, resumo em inglês, resumo em inglês, resumo em
inglês, resumo em inglês, resumo em inglês, resumo em inglês, resumo em inglês,
resumo em inglês, resumo em inglês, resumo em inglês, resumo em inglês, resumo
em inglês, resumo em inglês, resumo em inglês, resumo em inglês, resumo em
inglês, resumo em inglês, resumo em inglês, resumo em inglês, resumo em inglês,
resumo em inglês, resumo em inglês, resumo em inglês, resumo em inglês, resumo
em inglês,resumo em inglês. 
}

\Keyword{ % Palavras-Chaves em Inglês
Select up to five keywords to precisely define the content of your work.}



% Comando para geração da Capa
\Capa

% Comando para geração da Fohas de Rosto
\FolhaRosto

% TABLE OF CONTENTS -  OPTIONAL
\Sumario{3} % Gera índice com profundidade x

%%%%%%%%%%%%%%%%%%%%%%%%%%%%%%%%%%%%%%%%%%%%%%%%%%%%%%%%%%%%%
%%                                                         %%
%%          BEGIN DOCUMENT                                 %%
%%                                                         %%
%%%%%%%%%%%%%%%%%%%%%%%%%%%%%%%%%%%%%%%%%%%%%%%%%%%%%%%%%%%%%

\section{Seção}

Exemplo de citação \cite{article-full}, \cite{incollection-full},
\cite{techreport-full}.

Exemplo de referência para a subseção \ref{sec:subsec2}.

Exemplo de figura.

\begin{figure}[htb]
\begin{center}
 \includegraphics[width=0.5\textwidth,keepaspectratio,
bb= 14 14 1154 510]{./LogoPUC.jpg}
\caption{Exemplo de figura}
\label{fig:logo}
\end{center}
\end{figure}

Exemplo de referência à figura \ref{fig:logo}.

\subsection{Subseção}
Subseção.

\subsubsection{Sub subseção}
Sub Subseção.

Exemplo de nota de rodapé\footnote{Esta é uma nota de rodapé}, outra
nota\footnote{Outra nota de rodapé}.

\subsubsection{Sub subseção}
Sub Subseção.

Exemplo de lista sem numeração:
\begin{itemize}
 \item Item
  \begin{itemize}
   \item Subitem
   \item Subitem
   \item Subitem
  \end{itemize}
 \item Item
  \begin{itemize}
   \item Subitem
  \end{itemize}
\end{itemize}


\noindent % Para não identar o parágrafo
Exemplo de lista numerada:
\begin{enumerate}
 \item Item
  \begin{enumerate}
   \item Subitem
   \item Subitem
  \end{enumerate}
 \item Item
  \begin{enumerate}
   \item Subitem
   \item Subitem
   \item Subitem
  \end{enumerate}
\end{enumerate}



\subsubsection{Sub subseção}
Sub Subseção.

\subsection{Subseção}
Subseção.

\subsubsection{Sub subseção}
Sub Subseção.

\subsubsection{Sub subseção}
Sub Subseção.

\subsubsection{Sub subseção}
Sub Subseção.

\section{Seção}
Seção.

\subsection{Subseção}\label{sec:subsec2}
Subseção.

\subsubsection{Sub subseção}
Sub Subseção.

\subsubsection{Sub subseção}
Sub Subseção.

\subsubsection{Sub subseção}
Sub Subseção.

\subsection{Subseção}
Subseção.

\subsubsection{Sub subseção}
Sub Subseção.

\subsubsection{Sub subseção}
Sub Subseção.

\subsubsection{Sub subseção}
Sub Subseção.





%----------------------------------------------------------------------
\addcontentsline{toc}{section}{Referências} % Para incluir no sumário
\bibliographystyle{abnt-puc}
\bibliography{exemplo} % Nome do arquivo Bibtex (*.bib)


%----------------------------------------------------------------------
\appendix % a partir daqui todas as seções são tratadas como apendices
\newpage
\section{Apendice}
Seção.

\subsection{Subseção}
Subseção.

\subsubsection{Sub subseção}
Sub Subseção.


\section{Apendice}
Seção.

\subsection{Subseção}
Subseção.

\subsubsection{Sub subseção}
Sub Subseção.

\subsubsection{Sub subseção}
Sub Subseção.


%------------------------------------------------------------------------- 

\end{document}

 
